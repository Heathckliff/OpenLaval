\documentclass[a4j,10pt,oneside,openany]{jsbook}
%
\usepackage{amsmath,amssymb}
\usepackage{bm}
\usepackage{graphicx}
\usepackage{ascmac}
\usepackage{makeidx}
%
\makeindex
%
\newcommand{\diff}{\mathrm{d}}  %微分記号
\newcommand{\divergence}{\mathrm{div}\,}  %ダイバージェンス
\newcommand{\grad}{\mathrm{grad}\,}  %グラディエント
\newcommand{\rot}{\mathrm{rot}\,}  %ローテーション
%
\setlength{\textwidth}{\fullwidth}
\setlength{\textheight}{44\baselineskip}
\addtolength{\textheight}{\topskip}
\setlength{\voffset}{-0.6in}
%
\title{{\Huge \textbf{超音速衝動タービンの設計方法}}\\ {\small Ver. 0.1.0}}

\author{Arthur\\ \texttt{Supersonic Turbine Design}}
\date{\today}
\begin{document}
%
%
\maketitle
\frontmatter
\tableofcontents
%
%
\mainmatter

\chapter{設計手法について}
\begin{abstract}

このドキュメントは超音速衝動タービンの翼設計を行えるようにまとめたものである。
主に〜と〜による論文を参考にし、その設計方法に従う。
最後には設計を行えるプログラムとその使用方法について述べる。
はじめに断っておくが、本ドキュメントによる設計手法では決め打ちのパラメータが多々存在し、
最適解になるわけではないため、効率の良いタービンブレードを作成するためには、
実際に何個もブレードを試作し、動かし、効率良のかったタービンブレードのパラメータを
採用する方式を取らざるを得ない

\end{abstract}

\section{超音速衝動タービンの設計に当たって}
超音速衝動タービンの設計はタービン翼入り口で発生する衝撃波を打ち消すように設計するため、
必要とする変数は限られてくる。そのためタービン出力などのパラメータに左右されず翼設計が行える。

\section{マッハ角}
...
...

\section{もう、俺はダメだぁ}
...
...

\section{ブレードエッジの切り落とし}
...
...

\chapter{衝撃波打ち消し曲線}
\begin{abstract}
超音速タービンの翼設計で一番最初に行うのが衝撃波を打ち消す曲線を作ることである。
この作業が終わると別の翼形状は自動的に決まってくる。
\end{abstract}

\section{変数リスト}
基本的に超音速衝動タービンを設計するに当たって必要な物理パラメータは、
ガス比熱とガス入射角度のみである\\

ではリストアップする。

\begin{thebibliography}{20}
\bibitem{...}...
...
\end{thebibliography}

\newpage
\printindex
%
%
\end{document}
